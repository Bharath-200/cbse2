\documentclass[12pt,-letter paper]{article}                       
\usepackage{siunitx}                                              
\usepackage{setspace}
\usepackage{gensymb}                                              
\usepackage{xcolor}                                               
\usepackage{caption}
%\usepackage{subcaption}
\doublespacing                                                    
\singlespacing                                                    
\usepackage[none]{hyphenat}
\usepackage{amssymb}
\usepackage{relsize}
\usepackage[cmex10]{amsmath}
\usepackage{mathtools}
\usepackage{amsmath}                                              
\usepackage{commath}                                              
\usepackage{amsthm}
\interdisplaylinepenalty=2500
%\savesymbol{iint}
\usepackage{txfonts}                                              
%\restoresymbol{TXF}{iint}                                        
\usepackage{wasysym}                                              
\usepackage{amsthm}
\usepackage{mathrsfs}                                             
\usepackage{txfonts}                                              
\let\vec\mathbf{}
\usepackage{stfloats}
\usepackage{float}
\usepackage{cite}
\usepackage{cases}                                                
\usepackage{subfig}                                               
%\usepackage{xtab}
\usepackage{longtable}
\usepackage{multirow}
%\usepackage{algorithm}
\usepackage{amssymb}
%\usepackage{algpseudocode}
\usepackage{enumitem}
\usepackage{mathtools}
%\usepackage{eenrc}
%\usepackage[framemethod=tikz]{mdframed}                          
\usepackage{listings}                                             
%\usepackage{listings}
\usepackage[latin1]{inputenc}
%%\usepackage{color}{
%%\usepackage{lscape}
\usepackage{textcomp}
\usepackage{titling}
\usepackage{hyperref}
%\usepackage{fulbigskip}
\usepackage{tikz}
\usepackage{graphicx}                                             
\lstset{
  frame=single,
  breaklines=true
}
\let\vec\mathbf{}
\usepackage{enumitem}                                             
\usepackage{graphicx}                                             
\usepackage{siunitx}
\let\vec\mathbf{}                                                 
\usepackage{enumitem}
\usepackage{graphicx}
\usepackage{enumitem}
\usepackage{tfrupee}
\usepackage{amsmath}
\usepackage{amssymb}
\usepackage{mwe} % for blindtext and example-image-a in example
\usepackage{wrapfig}
\graphicspath{{/storage/self/primary/Download/latexnew/fig}}                                                \graphicspath{{figs/}}\providecommand{\mydet}[1]{\ensuremath{\begin{vmatrix}#1\end{vmatrix}}}               \providecommand{\myvec}[1]{\ensuremath{\begin{bmatrix}#1\end{bmatrix}}}\providecommand{\cbrak}[1]{\ensuremath{\left\{#1\right\}}}                                 \providecommand{\brak}[1]{\ensuremath{\left(#1\right)}}\graphicspath{{/storage/self/primary/Download/latexnew/fig}}  

\title{ CBSE Assignment 30-1-2}

\date{\today}


\begin{document}
\maketitle{}
\begin{enumerate}
     

\item Find the coordinates of a point $A$, where $AB$ is a diameter of the circle with centre $(-2, 2)$ and $B$ is the point with coordinates $(3, 4)$.
\item Find the value of $k$ for which the following pair of linear equations have infinitely many solutions. $2x+3y=7$ , $(k+1)x+(2k-1)y=4k+1$
\item Find the area of the segment shown in \figref{fig:figure1}, if radius of the circle is $21 cm$ and $\angle AOB = 120\degree$ Use $\brak{n =\frac{22}{7}} $
\begin{figure}[H]                                     
\centering
	
 \includegraphics[width=\columnwidth]{figs/img1.jpg}
		
\caption{}
		
\label{fig:figure1}
\end{figure}
\item In \figref{fig:figure2}, a circle is inscribed in a $\triangle ABC$ having sides $BC=8 cm$, $AB = 10cm$ and $AC = 12 cm$. Find the lengths $BL$, $CM$ and $AN$.
                                         
\begin{figure}[H]                                     
\centering
\includegraphics[width=\columnwidth]{figs/img2.jpg}
\caption{}
\label{fig:figure2}

 \end{figure}
\item prove that \begin{align} \frac{\tan^2A}{\tan^2 A-1}+\frac{\csc^2 A}{\sec^2 A-\csc^2 A}=\frac{1}{1-2\cos^2 A}\end{align}
\item The first term of an AP is 3, the last term is 83 and the sum of all its terms is 903. Find the number of terms and the common difference of the AP.
\item Construct a triangle $ABC$ with side $BC = 6 cm$, $\angle B=45^\degree, \angle A= 105^\degree$. Then construct another triangle whose sides are $\frac{3}{4}$ times the corresponding sides of the $\triangle ABC$

 
\item Two positive integers $a$ and $b$ can be written as $a = x^3*y^2$ and $b = x*y^3$. x,y are prime numbers. Find $LCM (a, b)$. 
\item If the sum of first $n$ terms of an $AP$ is $n^2$, then find its $10$th term.
\item Find all zeros of the polynomial $3x ^ 3 + 10x ^ 2 - 9x - 4$ if one of its zero is $1$.
\item Prove that $\frac{2+\sqrt{3}}{5}$  is an irrational number, given that $\sqrt{3}$ is an irrational number. 
\item if If $\sec\theta = x + \frac{1}{4x}$, where $x \neq 0$, find $(\sec\theta + \tan\theta)$.
\item Prove that the ratio of the areas of two similar triangles is equal to the square of the ratio of their corresponding sides.
\end{enumerate} 
\end{document}
